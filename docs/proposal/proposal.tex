%% The first command in your LaTeX source must be the \documentclass command.
\documentclass{acmart}

%% \BibTeX command to typeset BibTeX logo in the docs
\AtBeginDocument{%
  \providecommand\BibTeX{{%
    \normalfont B\kern-0.5em{\scshape i\kern-0.25em b}\kern-0.8em\TeX}}}

%% Need this to remove the awful DOI link in the reference
\acmDOI{}

%% Bibliography file
\bibliographystyle{acm}

%% end of the preamble, start of the body of the document source.
\begin{document}

%% The "title" command has an optional parameter,
%% allowing the author to define a "short title" to be used in page headers.
\title{Strategic multi-agent decision-making on Hong Kong Mahjong: a zero-sum, adverserial, hidden-information board game}

%% The "author" command and its associated commands are used to define
%% the authors and their affiliations.
\author{Bernardo Conde}
\email{bernardoconde@tecnico.ulisboa.pt}
\author{Pedro Carrott}
\email{pedro.carrott@tecnico.ulisboa.pt}
\author{Tiago Gonçalves}
\email{tiago.manuel.severino.goncalves@tecnico.ulisboa.pt}
\affiliation{%
  \institution{Instituto Superior Técnico, Universidade de Lisboa}
  \city{Lisboa}
  \country{Portugal}
}

\renewcommand{\shortauthors}{Conde, Carrott and Gonçalves}

%% The abstract is a short summary of the work to be presented in the article.
\begin{abstract}
  Board games have always been a very good environment to show complex multi-agent decision making, due to the existance of usually easy-to-define rules
  and a huge viable strategy space. One of the most analysed games for adverserial multi-agent decision making has been Poker.
  Unfortunately, Poker has a very high hidden information space, only allowing players to know about their cards, the river, and everyone's bets;
  making bluffing and deception strategies quite tricky and high-risk.
  We decided to model a different, more complex game: Hong Kong Mahjong. Due to its tile-discarding mechanic and complex scoring system\cite{hkmahjongrules},
  we assume that it will allow for more varied and more deceptive gamestyles to flourish. We will also implement different agents, that will atempt to compete
  with each other, in order to show the viability of these multiple strategies.
\end{abstract}


%% This command processes the author and affiliation and title information and builds the first part of the formatted document.
\maketitle

\section{Introduction}
Analysis of hidden-information multiplayer games allows us to better understand the dynamics between adverserial, multi-agent systems. In this paper, we present
an analysis of multiple strategies applied to the game of the Hong Kong variety of Mahjong. We recomend that the professors familiarize themselves with
the rules\cite{hkmahjongrules} before continuing, since the game is way too complex to fully explain in this proposal. In short: Hong Kong Mahjong is a 4-player,
turn-based game, where each of the 4 players tries to make a special hand to win the game. Each player has available to them 13 pieces, plus 1 more piece that it
draws from the Wall, a large struture comprising of the remaining game pieces. Each player is forced to discard one piece after drawing, showing it to every one
permanently. Instead of drawing, after a discard, any other player can also rob the discarded piece, if it completes a meld (special piece configuration).
If a meld is completed by robbing, this meld becomes open, visible to every other oponent. The round ends when one player announces that it has a winning hand.
As a reward for winning, a number of points, dependent of the value of the winning hand is transfered from one or more losing players to the winning one.
Each player starts with 20000 points and the game ends when one of the players gets to 0 or negative points. The player with most points wins.

\section{Approach}
In order to model Hong Kong Mahjong, we first needed to write our own simulator of it. This simulator, written in Java, was appropriately called MahJava. 
It follows a very simple Client-Server architecture: there are 4 clients, called the Players, that join a central Server. When these 4 Players are present,
the server starts a new game, keeping then track of each of the Players' interactions, and giving them the necessary information. All these links between players and
servers need to go ``both ways'', since both need to be able to trade information with each other. Because both the Players and the Server need to keep track of the
game state, we have isolated this shared functionality onto a shared library.
To deal with all the comunications between Players and the Server, we are using the Spring Boot framework\cite{springbootfw} and its WebSocket\cite{mdnwebsocket}
functionality, in order to guarantee reliable, 2-way comunication.
Each Player will then have a special Profile, a strategy that it employs during the game. This strategy can either be as simple as probability analysis of the remaining
tiles, or as complicated as a fully-trained Deep Learning model.

\section{Evaluation}
Some interesting metrics of evaluation would be:
\begin{itemize}
\item Percentage of wins/losses/draws, total and against every other Profile
\item Balance of points, total and against every other Profile
\item Variance/Deviation in Profile matchups
\item etc
\end{itemize}

\bibliography{biblio}
\end{document}
\endinput
